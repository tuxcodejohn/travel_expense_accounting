%%%%%%%%%%%%%%%%%%%%%%%%%%%%%%%%%%%%%%%%%%%%%%%%%%%%%%%%%%%%%%%%%
%% Header Area

\documentclass[a4paper,10pt]{scrartcl}

\usepackage{a4wide}
\usepackage[ngerman]{babel}
%\usepackage[T1]{fontenc}
%\usepackage[utf8]{luainputenc}
\usepackage{fontspec}
\usepackage{textcomp}
\usepackage[table]{xcolor}
\usepackage{hhline}
\usepackage{tabularx}
\usepackage{tabu}
\usepackage{longtable}
\usepackage{multirow}
\usepackage[yyyymmdd]{datetime}
\usepackage{numprint}
\usepackage{luacode}


%\usepackage{showframe}

% modern style (you have it in lualatex!) of 
% setting fonts:
\setmainfont[Scale=.8]{Comic Sans MS}
\setmainfont[Scale=1]{Linux Biolinum O}
\setmonofont{Courier Prime}

%% for company logo:
\usepackage{tikz}
\usepackage{graphicx}

\pgfdeclareimage[height=2cm]{Companylogo}{logo.pdf}

%%

\definecolor{OrtGray}{gray}{0.75}
\newcommand{\spcell}[2][c]{%
	\begin{tabular}[#1]{@{}c@{}}#2\end{tabular}}

\setlength\LTleft{0pt}
\setlength\LTright{0pt}

\nprounddigits{2}

\renewcommand{\dateseparator}{-}

%%%%%%%%%%%%%%%%%%%%%%%%%%%%%%%%%%%%%%%%%%%%%%%%%%%%%%%%%%%%%%%%%
%% Configuration Area
\def \TravHeading   { Reisekostenabrechnung }
\def \TravName      { Handels Reisender } % Travelers name
\def \TravTarget    { Münchhausen }       % Travelers target
\def \TravStartedOn { \formatdate{01}{01}{2014} } % date of start
\def \TravStartedAt { \formattime{08}{45}{00} }   % time of start
\def \TravEndedOn   { \formatdate{05}{01}{2014} } % date of end
\def \TravEndedAt   { \formattime{18}{15}{00} }   % time of end
\def \TravReason    { Inbetriebnahme Automatisierung. Kann ein sehr langer Text sein. }
\def \TravProjNr    { PRJ001 }

\begin{luacode*}
	NrShortDays  = 2 -- >8h <24h
	NrFullDays   = 3 -- =24h
	NrBreakfasts = 4 -- Nr of days you got free breakfast
	NrLunches    = 1 -- Nr of days you got free lunch

	MoneyInAdvance = 150.00

	-- night stops
	NightStopCurrency        = '\\texteuro'
	NightCurrencyTranslation = 1 -- from currency to Euro
	NightStops = {
		-- nr of nights, description              | date                        | orig. currency
		{ 1            , "Landgasthof Ungemütlich", "\\formatdate{03}{01}{2014}", 79.20          },
		{ 2            , "Zum Wanderer"           , "\\formatdate{04}{01}{2014}", 50.87          },
		{ 1            , "Hotel Garni Lang"       , "\\formatdate{06}{01}{2014}", 40.00          },
	}

	-- private car
	PrivateCar = {
		-- distance(km), description             | date                         | mileage rate (0.30 Euro/km for private cars)
		{ 500          , "Anfahrt Fabrik"        , "\\formatdate{03}{01}{2014}", 0.30          },
		{ 37           , "Einkaufen"             , "\\formatdate{04}{01}{2014}", 0.30          },
		{ 121          , "Verfolgungsjagt"       , "\\formatdate{06}{01}{2014}", 0.30          },
	}

	-- other expenses
	OtherExpensesCurrency            = '\\texteuro'
	OtherExpensesCurrentyTranslation = 1 -- from currency to Euro
	OtherExpenses = {
		-- company     , description               | date                        | orig. currency
		{ "Taxi Müller", "Münchhausen -> Flughafen", "\\formatdate{03}{01}{2014}", 50.00          },
	}

	---------------------------------------------------------
	-- Functions - no configuration, stop looking
	
	function PrintTable (DataArray)
		for i = 1, #DataArray do
			tex.sprint(tostring(i), '&', DataArray[i][1], '&', DataArray[i][2], '&', DataArray[i][3], '&', "\\numprint{", DataArray[i][4] , "}", '&',  "\\numprint{", (NightCurrencyTranslation * DataArray[i][4]) , "}", "\\\\ ")
			if i ~= #DataArray then
				tex.sprint("\\tabucline{1-5}")
			else
				tex.sprint("\\hline")
			end
		end
	end

	function SumExpenses (DataArray)
		expenses = 0
		for i = 1, #DataArray do
			expenses = expenses + (NightCurrencyTranslation * DataArray[i][4])
		end
		return expenses
	end

	function PrintTablePrivCar (DataArray)
		for i = 1, #DataArray do
			tex.sprint(tostring(i), '&', DataArray[i][1], '&', DataArray[i][2], '&', DataArray[i][3], '&', "\\numprint{", DataArray[i][4] , "}", '&',  "\\numprint{", (DataArray[i][1] * DataArray[i][4]) , "}", "\\\\ ")
			if i ~= #DataArray then
				tex.sprint("\\tabucline{1-5}")
			else
				tex.sprint("\\hline")
			end
		end
	end

	function SumExpensesPrivCar (DataArray)
		expenses = 0
		for i = 1, #DataArray do
			expenses = expenses + (DataArray[i][1] * DataArray[i][4])
		end
		return expenses
	end

	function AllExpenses ()
		return SumExpenses(NightStops) + SumExpensesPrivCar(PrivateCar) + SumExpenses(OtherExpenses) - MoneyInAdvance 
	end
\end{luacode*}

%%%%%%%%%%%%%%%%%%%%%%%%%%%%%%%%%%%%%%%%%%%%%%%%%%%%%%%%%%%%%%%%%
%% Fees

\directlua {
	FeeShortDay       = 12
	FeeFullDay        = 24
	FeePercBreakfast  = 0.2
	FeePercLunch      = 0.4
}

%%%%%%%%%%%%%%%%%%%%%%%%%%%%%%%%%%%%%%%%%%%%%%%%%%%%%%%%%%%%%%%%%
%% Caculations

\directlua {
	FeeBreakfast = -1 * FeeFullDay * FeePercBreakfast
	FeeLunch     = -1 * FeeFullDay * FeePercLunch


	AbsoluteShortDays  = FeeShortDay * NrShortDays
	AbsoluteFullDays   = FeeFullDay * NrFullDays
	AbsoluteBreakfasts = FeeBreakfast * NrBreakfasts
	AbsoluteLunches    = FeeLunch * NrLunches
	SumFare            = AbsoluteShortDays + AbsoluteFullDays + AbsoluteBreakfasts + AbsoluteLunches
}

%%%%%%%%%%%%%%%%%%%%%%%%%%%%%%%%%%%%%%%%%%%%%%%%%%%%%%%%%%%%%%%%%
%% Document Meta Information

\usepackage[pdftex,
	citebordercolor={0.9 0.9 1},
	filebordercolor={0.9 0.9 1},
	linkbordercolor={0.9 0.9 1},
	pagebordercolor={0.9 0.9 1},
	urlbordercolor={0.9 0.9 1},
	pdfborder={0.9 0.9 1},
	pagebackref,plainpages=false,pdfpagelabels=true]{hyperref}
\hypersetup{%
pdftitle={Reisekostenaberechnung \TravName},
pdfauthor={\TravName},
pdfkeywords={Reisekosten, Abrechnung, \TravName}, 
pdfsubject={Reisekosten},
}
%%%%%%%%%%%%%%%%%%%%%%%%%%%%%%%%%%%%%%%%%%%%%%%%%%%%%%%%%%%%%%%%%
%% Document Area

\author{\TravName}
\date{\today}

\begin{document}
%\Large{\TravHeading}

\begin{tikzpicture}[remember picture, overlay]
  \node [anchor=north east, inner sep=0pt]  at (current page.north east)
     {\pgfuseimage{Companylogo}};
\end{tikzpicture}

\begin{table}
	\begin{tabularx}{\textwidth}{lXlX}
	\textbf{Reisender:}                  & \multicolumn{3}{l}{ \TravName }                     \\
	\textbf{Reiseziel:}                  & \multicolumn{3}{l}{ \TravTarget }                   \\
	\textbf{Reisebeginn:}                & \TravStartedOn & \textbf{Uhrzeit:} & \TravStartedAt \\
	\textbf{Reiseende:}                  & \TravEndedOn   & \textbf{Uhrzeit:} & \TravEndedAt   \\
        \textbf{Reisezweck:}                 & \multicolumn{3}{l}{ \TravReason }                   \\
	\textbf{Verkaufsauftr./Projekt-Nr.:} & \multicolumn{3}{l}{ \TravProjNr }
	\end{tabularx}
\end{table}

\begin{longtabu} to 0.94\linewidth {|c|c|X|c|c|c|} % FIXME why is \linewidth or \textwidth bigger than the page???

\hline

\multicolumn{6}{|l|}{\textbf{Verpflegung pro Kalendertag}} \\
\hline
\taburowcolors 1 {OrtGray .. OrtGray}
~ & Anzahl Tage   & Pauschale              & \spcell{Pauschale \\ {[\texteuro]}} & ~ & \spcell{Betrag \\ {[\texteuro]}}	\\ \hline
\taburowcolors 1 {white .. white}
~ & \directlua{ tex.sprint(NrShortDays) }  & 8:01 -- 23:59 Stunden               & \directlua{ tex.sprint(FeeShortDay) }  & ~ & \numprint{\directlua{ tex.sprint(AbsoluteShortDays) }}              \\ \tabucline{1-5}
~ & \directlua{ tex.sprint(NrFullDays) }   & 24 Stunden                          & \directlua{ tex.sprint(FeeFullDay) }   & ~ & \numprint{\directlua{ tex.sprint(AbsoluteFullDays) }}               \\ \tabucline{1-5}
~ & ~             & ~                      & ~                                   & ~                                      & ~                                                         \\ \tabucline{1-5}
~ & \directlua{ tex.sprint(NrBreakfasts) } & Frühstück                           & \numprint{\directlua{ tex.sprint(FeeBreakfast) }} & ~ & \numprint{\directlua{ tex.sprint(AbsoluteBreakfasts) }}             \\ \tabucline{1-5}
~ & \directlua{ tex.sprint(NrLunches) }    & Mittagessen                         & \numprint{\directlua{ tex.sprint(FeeLunch) }}     & ~ & \numprint{\directlua{ tex.sprint(AbsoluteLunches) }}                \\ 
\hline
\taburowcolors 1 {OrtGray .. OrtGray}
\multicolumn{3}{|l}{\parbox[t]{7cm}{Der VMA wird nach Prüfung mit der nächsten Gehaltsabrechnung überwiesen.}} & \multicolumn{1}{r}{ \textbf{Summe VMA:} } & \multicolumn{2}{r|}{ \textbf{ \numprint{\directlua{ tex.sprint(SumFare) }} \texteuro} } \\ 
\taburowcolors 1 {white .. white}

\hline

\multicolumn{6}{l}{ ~ } \\

\hline
\multicolumn{6}{|l|}{\textbf{Übernachtung}} \\
\hline
\taburowcolors 1 {OrtGray .. OrtGray}
\# & Anzahl & Beschreibung & Datum & \spcell{Urspr.währung \\ {[}\directlua{ tex.sprint(NightStopCurrency) }{]} } & \spcell{Betrag \\ {[\texteuro]}} \\ \hline
\taburowcolors 1 {white .. white}
\directlua{
	PrintTable(NightStops)
}

\multicolumn{6}{|l|}{\textbf{Fahrtkosten privater PKW}} \\
\hline
\taburowcolors 1 {OrtGray .. OrtGray}
\# & \spcell{Strecke \\ {[km]}} & Beschreibung & Datum & \spcell{Km Satz \\ {[\texteuro]/km}} & \spcell{Betrag \\ {[\texteuro]}} \\ \hline
\taburowcolors 1 {white .. white}
\directlua{
	PrintTablePrivCar(PrivateCar)
}

\multicolumn{6}{|l|}{\textbf{Weitere Belege}} \\
\hline
\taburowcolors 1 {OrtGray .. OrtGray}
\# & Firma & Beschreibung & Datum & \spcell{Urspr.währung \\ {[}\directlua{ tex.sprint(OtherExpensesCurrency) }{]} } & \spcell{Betrag \\ {[\texteuro]}} \\ \hline
\taburowcolors 1 {white .. white}
\directlua{
	PrintTable(OtherExpenses)
}

\hline

\hline
\hline


\taburowcolors 1 {OrtGray .. OrtGray}
\multicolumn{6}{|l|}{ ~ } \\
\multicolumn{4}{|r}{ \textbf{Summe Übernachtungen} } & \multicolumn{2}{r|}{ \textbf{ \numprint{\directlua{ tex.sprint(SumExpenses(NightStops)) }} \texteuro} } \\
\multicolumn{4}{|r}{ \textbf{Summe Fahrtkosten Privat PKW} } & \multicolumn{2}{r|}{ \textbf{ \numprint{\directlua{ tex.sprint(SumExpensesPrivCar(PrivateCar)) }} \texteuro} } \\
\multicolumn{4}{|r}{ \textbf{Summe Belege} } & \multicolumn{2}{r|}{ \textbf{\numprint{\directlua{ tex.sprint(SumExpenses(OtherExpenses)) }} \texteuro} } \\
\multicolumn{4}{|r}{ \textbf{Vorschuss} } & \multicolumn{2}{r|}{ \textbf{ \numprint{\directlua{ tex.sprint(MoneyInAdvance) }} \texteuro} } \\ \tabucline[OrtGray]{1-3}\noalign{\kern-\arrayrulewidth} \tabucline{4-6}
\multicolumn{4}{|r}{ \textbf{Summe Belege} } & \multicolumn{2}{r|}{ \textbf{ \numprint{\directlua{ tex.sprint(AllExpenses()) }} \texteuro} } \\ \tabucline[OrtGray]{1-3}\noalign{\kern-\arrayrulewidth} \tabucline{4-6}
\multicolumn{6}{|l|}{ Die Summe der Belege wird nach Prüfung beim nächsten Überweisungstermin angewiesen. } \\

\hline
\taburowcolors 1 {white .. white}
\multicolumn{6}{l}{ ~ } \\
\hline

\taburowcolors 1 {OrtGray .. OrtGray}
\multicolumn{6}{|l|}{ ~ } \\
\multicolumn{3}{|r}{ \textbf{Bestätigung der Summe durch den Reisenden} } & \multicolumn{3}{r|}{ ~ } \\ \tabucline[OrtGray]{1-3}\noalign{\kern-\arrayrulewidth} \tabucline{4-6}
\multicolumn{3}{|r}{ ~ } & \multicolumn{1}{l}{ \emph{Ort/Datum} } & \multicolumn{2}{r|}{ \emph{Unterschrift} } \\
\multicolumn{6}{|l|}{ ~ } \\
\hline
\end{longtabu}

\end{document}
